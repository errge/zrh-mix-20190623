%% -*- latex -*-
\newtoks\nicsroot\directlua{ tex.settoks("nicsroot", os.getenv("NICS_ROOT") or error("NICS_ROOT environment variable has to be set, use the Makefile")) }
\input{\the\nicsroot /src/nics-cached.tex}
\endofdump
\input{\the\nicsroot /src/nics-noncached.tex}
\hypersetup{
  pdfauthor={your name or emailss},
  pdftitle={your presentation title},
}
\nicsgrid=0
\begin{document}

\section{Nilcons vegyes}

\nicstitleslide{images/welcome}{ZRH Gergő 2019-06-23}{Mixed stuff}

\begin{slide}{Agenda}{}
  \begin{nicscolumn}
    \nicsitem{Bootstrap SCSS: web design for nics webpage}
    \nicsitem{Nilcons Fonts: usable Linux fonts independently from your system}
    \nicsitem{Saját font: Jozsika, yay!}
    \nicsitem{Virtual Machine update: nested VMs, licensing, ddccontrol}
    \nicsitem{Remote working: IPsec + nx + usbip for yubikey, ykselect}
    \nicsitem{xrandr érdekesség: monitor absztrakció}
    \nicsitem{Linux érdekesség: vrf absztrakció}
    \nicsitem{IPsec + UDP hole punching with someone}
  \end{nicscolumn}
\end{slide}

\begin{slide}{Bootstrap SCSS}{Web design for nics}
  \begin{nicscolumn}
    \nicspar{\centering\url{https://github.com/nilcons/nics/tree/gh-pages}}
    \nicsbigskip
    \nicsitem{Érdekességek megmutatni}
    \begin{nicsindent}
      \nicsitem{\mono{package.json}}
      \nicsitem{\mono{gulpfile.js}}
      \nicsitem{\mono{index.html} + \mono{nics.scss}}
      \begin{nicsindent}
        \nicsitem{a struktúrát diktálja a bootstrap doksi}
        \nicsitem{grid: 12 oszlop soronként, mert az mindennel osztható}
        \nicsitem{responsive: col-\{sm,md,lg,xl\}-4}
        \nicsitem{miért sokkal jobb így használni a bootstrapet?}
      \end{nicsindent}
    \end{nicsindent}
  \end{nicscolumn}
\end{slide}

\begin{slide}{Nilcons Fonts}{Independent fonts from your buggy GNU/Linux distro}
  \begin{nicscolumn}
    \nicspar{\centering\url{https://github.com/nilcons/nilcons-fonts}}
    \nicsbigskip
    \nicsitem{Érdekes fájlok}
    \begin{nicsindent}
      \nicsitem{\mono{README.md} teteje}
      \nicsitem{\mono{ncfonts.conf}}
      \nicsitem{\mono{fonts/ könyvtár}}
      \nicsitem{\mono{conf.d/10-nilcons-base.conf}}
    \end{nicsindent}
    \nicsbigskip
    \nicsitem{5 hónapja daily driver for both of us}
  \end{nicscolumn}
\end{slide}

\begin{slide}{Iosevka}{Saját kiadás: Józsika}
  \begin{nicscolumn}
    \nicspar{\centering\url{https://github.com/be5invis/Iosevka}}
    \nicspar{\centering\url{https://github.com/nilcons/jozsika}}
    \nicsbigskip
    \nicsitem{Az Iosevka egy zseniális cucc, érdemes tudni róla!}
    \nicsitem{De a közelebb tolást (amit én szerettem volna) nem tudta}
    \nicsitem{De könnyű volt beletenni, habár hackyn, magamnak jó lesz!}
    \nicsitem{Terminálokban és Emacsban daily driver}
    \nicsitem{Kb. 1.5x annyi betűt hívok olvashatónak az új fonttal, mint régen}
  \end{nicscolumn}
\end{slide}

\begin{slide}{Gergo Tech update}{Virtuális gépek for gaming and saving, microsoft licensing, ddccontrol}
  \begin{nicscolumn}
    \nicspar{\centering\url{https://github.com/nilcons/windows-usability}}
    \nicsbigskip
    \nicsitem{Previously on Gergo Tech}
    \begin{nicsindent}
      \nicsitem{Virtuális gép win gamingre PCI video passhtroughval (no FPS loss)}
      \nicsitem{PCI video passhtrough VM munkaállomásnak szerveren}
      \nicsitem{Microsoft licensing: olcsó windows és office}
    \end{nicsindent}
    \nicsitem{Update: mindez működik, nem csalódtam, semmit nem ,,tiltottak'' le}
    \nicsitem{Update: ddccontrol}
    \begin{nicsindent}
      \nicsitem{Váltani is kellene a monitoron az inputot a windowsra}
      \nicsitem{Ez automatizálható ddccontrollal}
      \nicsitem{Megy vezérlőjel a HDMI/DisplayPort/DVI kábelen, tudjunk róla}
      \nicsitem{Minden másra is használható: brightness, volume, stb.}
    \end{nicsindent}
  \end{nicscolumn}
\end{slide}

\begin{slide}{Remote working}{NX, IPsec, usbip for yubikey, ykselect}
  \begin{nicscolumn}
    \nicsitem{NX: good, IPsec: good}
    \nicsitem{NX has usb forward, but ...}
    \nicsitem{\mono{ykselect.sh}}
    \nicsitem{\mono{ykbind}}
  \end{nicscolumn}
\end{slide}

\begin{slide}{xrandr érdekesség}{Monitor absztrakció az xrandrban}
  \begin{nicscolumn}
    \nicsitem{\mono{docs/info/xrandr-monitors-outputs}}
    \nicsitem{\mono{xrandr --listmonitors}}
    \nicsitem{\mono{xrandr --setmonitor}}
  \end{nicscolumn}
\end{slide}

\begin{slide}{Linux érdekesség}{vrf absztrakció}
  \begin{nicscolumn}
    \nicsitem{\mono{ip vrf}: kernelben van több routing domain}
    \nicsitem{ezeket aztán cgrouphoz lehet kötni}
    \nicsitem{a cgrouphoz meg processzeket}
    \nicsitem{use case: a vpn tunnel-t csak a xyz app használhatja}
  \end{nicscolumn}
\end{slide}

\begin{slide}{IPsec + UDP hole punching with someone}{Let's make a tunnel between two machines without any public IP}
  \begin{nicscolumn}
    \nicspar{\centering\url{https://github.com/nilcons/ipsec-stun-explain}}
    \nicsitem{STUN: súg neked, hogy tudd hogyan működik a NAT-od}
    \nicsitem{UDP hole punching: két NAT mögötti gép is tud beszélni}
    \nicsitem{IPsec without any daemons: kernel side of IPsec}
  \end{nicscolumn}
\end{slide}

\end{document}
